\documentclass[
  a4paper,
  12pt
]{cv}


\renewcommand{\DatestampYMD}[3]{\mbox{#1年\number#2月}}

% \usepackage{ctex}
\usepackage{xeCJK}
% \setCJKmainfont{Heiti TC}
\author{富律文}
\contacts{ \href{mailto:lyuwen.fu@columbia.edu}{lyuwen.fu@columbia.edu} \,\SubBulletSymbol\, +86 180-7290-1557 \,\SubBulletSymbol\, \href{https://lyuwenfu.me}{lyuwenfu.me} }


\begin{document}

\maketitle

%% EDUCATION
\section{教育经历}

{\textbf{Columbia University}},
New York, New York

Ph.D. in
{Materials Science}
\hfill
\DatestampYMD{2017}{01}{10} --
\DatestampYMD{2021}{10}{20}
\begin{itemize}
\item 导师: Professor Chris Marianetti
\item 研究内容: 材料第一性原理计算,材料声子及声子相互作用的算法设计
\item 毕业论文: Thermodynamics of Interacting Phonons (\href{https://doi.org/10.7916/d8-wkbr-m336}{doi:10.7916/d8-wkbr-m336})
\end{itemize}


{\textbf{Columbia University}},
New York, New York

Master of Science in
{Materials Science}
\hfill
\DatestampYMD{2015}{09}{01} --
\DatestampYMD{2016}{12}{23}
\begin{itemize}
\item GPA: 3.81 / 4.00
\item 所学课程: 固体物理,复杂材料电子结构的计算,晶体的群论,材料机械性能,等
\end{itemize}


{\textbf{北京科技大学}},
北京, 中国

{材料物理} 工学学士
\hfill
\DatestampYMD{2011}{09}{01} --
\DatestampYMD{2015}{06}{26}
\begin{itemize}
\item 导师: 滕蛟教授
\item 毕业论文: 阻变存储器的量子输运的研究
\item GPA: 3.46 / 4.00
\end{itemize}


%% RESEARCH
\section{科研经历}

{\textbf{Columbia University}},
New York, New York

Department of Applied Physics and Applied Mathematics
\hfill
\DatestampYMD{2016}{05}{17} --
\DatestampYMD{2021}{10}{20}
\begin{itemize}
\item 科研项目: Thermodynamics of interacting phonons
\item 指导教授: Professor Chris Marianetti
\item 开发计算材料声子及声子间相互作用和预测材料电声子性能的算法和软件(代码体量约2万行);
\item 实现相比于现有同类软件超过10倍的计算效率提升;
\item 设计出可广泛应用在多种材料性能研究的模块化软件包;
\item 在高性能计算机集群上执行超过10万次第一性原理计算作业并管理课题组内超过80个节点的计算机集群;
\item 软件包主页:\href{https://marianettigroup.github.io}{marianettigroup.github.io}。
% \item Derived equation for order $N$ volume derivative of phonons in terms of order $\mathcal{N}=N+2$ irreducible derivatives; the former are easily computed,
% and serve as a stringent test for the latter.
\end{itemize}

{\textbf{北京科技大学}},
北京, 中国

%\item
本科生毕业设计, 
材料物理与化学学院
\hfill
\DatestampYMD{2015}{02}{01} --
\DatestampYMD{2015}{06}{25}
\begin{itemize}
\item 指导教授: 滕蛟教授;
\item 课题: 研究阻变存储器的量子输运性能; 
\item 设计薄膜阻变存储器的制备工艺,实现阻变存储器的阻变效应; 
\item 研究阻变存储器的原理及理论背景,以及量子各向异性磁电阻效应的实现条件; 
\item 研究、测试制成阻变存储器样品的阻变效应以及电阻输运性能。
\end{itemize}

%{\textbf{University of Science and Technology Beijing}},
%Beijing, China

\vspace{.25em}
本科生科研项目,
新材料技术研究院
\hfill
\DatestampYMD{2013}{10}{05} --
\DatestampYMD{2014}{05}{31}
\begin{itemize}
\item 课题: 一维IrO$_{2}$纳米阵列传感器的电化学性能的研究; 
\item 指导教授: 孟惠民教授; 
\item 开发一维IrO$_{2}$纳米阵列的新制备方法并研究电极的电化学性能。
\end{itemize}


\section{技能总结}
\begin{itemize}
\item
编程语言:   
Proficient in
Python, C/C++, 
Familiar with Objective-C, Java, C\#, Fortran.
%HTML, 
\item
第一性原理计算: 
VASP, Quantum ESSRESSO, Abinit.
\item
其他软件:    
{\LaTeX}, 
Docker,
MATLAB, 
Mathematica, 
Blender,
Adobe Photoshop.
\end{itemize}


\section{研究重心}
第一性原理计算和模拟;
计算材料科学;
声子和声子间相互作用;
材料热力学及热传导性能。


\section{发表文章}

\begin{enumerate}
\item
\href{https://doi.org/10.1103/PhysRevB.100.014303}
{\underline{Fu, L.}, Kornbluth, M., Cheng, Z., \& Marianetti, C. A. (2019). 
Group theoretical approach to computing phonons and their interactions. 
\textit{Physical Review B}, 100(1), 014303.}

第一作者, SCI-2区, 影响因子: 4.036, 25页, 入选 Editors' Suggestion
%
\item
\href{https://doi.org/10.1038/s42005-020-00483-2}{
Bryan, M. S., \underline{Fu, L.}, et al. (2020). 
Nonlinear propagating modes beyond the phonons in \\fluorite-structured crystals.
\textit{Communications Physics}, 3(1), 1-7.}

第二作者, SCI-1区, 影响因子: 6.368, 7页
%
\item
\href{https://doi.org/10.1080/09500839.2020.1833375}{
Ding, X., Yao, T., \underline{Fu, L.}, et al. (2020).
Magnetic, transport and thermal properties of $\delta$-phase UZr$_{2}$.
\textit{Philosophical Magazine Letters}, 1-11.}

第三作者, SCI-4区, 影响因子: 0.980, 11页
%
\item
\href{https://doi.org/10.1016/j.actamat.2021.116934}{
C.A. Dennett, \dots, \underline{L. Fu}, et al. (2021).
An Integrated Experimental and Computational Investigation of Defect and Microstructural Effects on Thermal Transport in Thorium Dioxide, %
\textit{Acta Mater.}, 213, 116934.
}

第七作者, SCI-1区, 影响因子: 8.203, 13页
%
% \item
% \href{https://arxiv.org/abs/1903.05253}{
% Wieteska, A., Foutty, B., Guguchia, Z., Flicker, F., Mazel, B., \underline{Fu, L.}, ... \& Pasupathy, A. (2019).
% Uniaxial Strain Tuning of Superconductivity in 2$H$-NbSe$_{2}$.
% \textit{arXiv preprint} arXiv:1903.05253.}
\end{enumerate}


\section{待发表文章}

\begin{enumerate}
\item
\href{https://arxiv.org/abs/2202.11041}{
E. Xiao, H. Ma, M. S. Bryan, \underline{L. Fu}, et al. (2022).
Validating First-Principles Phonon Lifetimes via Inelastic Neutron Scattering,
arXiv:2202.11041.
}

\item
\href{https://arxiv.org/abs/2202.14016}{
M. A. Mathis, A. Khanolkar, \underline{L. Fu}, et al. (2022).
The Generalized Quasiharmonic Approximation via Space Group Irreducible Derivatives,
arXiv:2202.14016.
}
\end{enumerate}


\section{会议演讲}

\begin{enumerate}
\item
\href{https://meetings.aps.org/Meeting/MAR18/Event/322388}{
Fu, L., Kornbluth, M., \& Marianetti, C. A. (2018).
An optimal approach to computing phonons and their interactions via finite difference.
APS March Meeting 2018, X29.00006.
}
%
\item
\href{https://meetings.aps.org/Meeting/MAR19/Session/H22.3}{
Fu, L., Kornbluth, M., Cheng, Z., \& Marianetti, C. A. (2019).
An optimal approach to computing phonons and their interactions via finite displacements.
APS March Meeting 2019, H22.00003.
}
%
\item
\href{https://meetings.aps.org/Meeting/MAR20/Session/P44.9}{
Fu, L., Mathis, M., Xiao, E., \& Marianetti, C. A. (2020).
Phonon interactions in rock salt and fluorite structures.
APS March Meeting 2020, P44.00009.
} (\emph{会议由于新冠疫情取消})
\end{enumerate}

\end{document}
