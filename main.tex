\documentclass[
  a4paper,
  12pt
]{cv}

\author{Lyuwen Fu}
\contacts{ \href{mailto:lyuwenfu@zhejianglab.edu.cn}{lyuwenfu@zhejianglab.edu.cn} \,\SubBulletSymbol\, +86 180-7290-1557 \,\SubBulletSymbol\, \href{https://lyuwenfu.me}{lyuwenfu.me} }


\begin{document}

\maketitle

%% EXPERIENCE
\section{Experience}

{\textbf{Zhejiang Lab}},
Hangzhou, Zhejiang

Senior Research Scientist in
{Center for Intelligent Computing}
\hfill
\DatestampYMD{2022}{07}{12} --
Present

%% EDUCATION
\section{Education}

{\textbf{Columbia University}},
New York, New York

Ph.D. in
{Materials Science}
\hfill
\DatestampYMD{2017}{01}{10} --
\DatestampYMD{2021}{10}{20}
\begin{itemize}
\item Advisor: Professor Chris Marianetti
\item Research Focus: Generic first-principle computation on phonons and phonon-phonon interactions
\item Thesis: Thermodynamics of Interacting Phonons
\end{itemize}


{\textbf{Columbia University}},
New York, New York

Master of Science in
{Materials Science}
\hfill
\DatestampYMD{2015}{09}{01} --
\DatestampYMD{2016}{12}{23}
\begin{itemize}
\item Cumulative GPA: 3.81 / 4.00
\item Coursework: Solid State Physics, Computing Electronic Structure of Complex Materials, Theory of Crystalline Materials, Mechanical Behavior of Materials, etc.
\end{itemize}


{\textbf{University of Science and Technology Beijing}},
Beijing, China

Bachelor of Engineering in
{Materials Physics}
\hfill
\DatestampYMD{2011}{09}{01} --
\DatestampYMD{2015}{06}{26}
\begin{itemize}
\item Adviser: Professor Jiao Teng
\item Thesis: Research on the Quantum Transport Properties of Resistive RAM
\item Cumulative GPA: 3.46 / 4.00
\end{itemize}


%% RESEARCH
\section{Research}

{\textbf{Columbia University}},
New York, New York

Department of Applied Physics and Applied Mathematics
\hfill
\DatestampYMD{2016}{05}{17} --
\DatestampYMD{2021}{10}{20}
\begin{itemize}
\item Project: Thermodynamics of interacting phonons
\item Adviser: Professor Chris Marianetti
\item Develop group theoretical approach to extract arbitrary order phonons and their interactions in terms of space group \emph{irreducible derivatives}.
\item Develop finite difference algorithm which extracts all irreducible derivatives in the smallest possible supercells with the fewest possible calculations.
\item Develop software to compute thermal dynamic properties using the extracted phonon interaction data.
\item Perform high-throughput computations on HPC clusters.
\item Manage an in-house cluster of more than 80 nodes.
% \item Derived equation for order $N$ volume derivative of phonons in terms of order $\mathcal{N}=N+2$ irreducible derivatives; the former are easily computed,
% and serve as a stringent test for the latter.
\end{itemize}

{\textbf{University of Science and Technology Beijing}},
Beijing, China

%\item
Undergraduate Thesis,
Department of Materials Physics and Chemistry
\hfill
\DatestampYMD{2015}{02}{01} --
\DatestampYMD{2015}{06}{25}
\begin{itemize}
\item Project: Research on the Quantum Transport Properties of Resistive RAM
\item Adviser: Professor Jiao Teng
\item Design the synthesis process and the pattern of ReRAM thin film. Realized the resistive switch phenomenon in ReRAM.
\item Research on the theoretical background of ReRAM and the mechanism and conditions for the Quantized Anisotropic Magnetoresistance.
\item Study the resistive switch properties and electron transport properties of the ReRAM samples.
\end{itemize}

%{\textbf{University of Science and Technology Beijing}},
%Beijing, China

\vspace{.25em}
Undergraduate Student Research, 
Institute for Advanced Materials and Technology
\hfill
\DatestampYMD{2013}{10}{05} --
\DatestampYMD{2014}{05}{31}
\begin{itemize}
\item Project: Research on the Performance Study of One-dimensional IrO$_{2}$ Nano-Array Electrochemical Sensor
\item Adviser: Professor Huimin Meng
\item Develop a new method of the synthesis of the one-dimensional IrO$_{2}$ nanometer array electrodes and studied electrochemistry properties of the electrode.
\end{itemize}


\section{Skills}
\begin{itemize}
\item
Programming Languages:   
Proficient in
Python, C/C++, 
Familiar with Objective-C, Java, C\#, Fortran.
%HTML, 
\item
First-principle Computation:   
VASP, Quantum ESSRESSO, Abinit.
\item
Other Softwares:    
{\LaTeX}, 
Docker,
MATLAB, 
Mathematica, 
Blender,
Adobe Photoshop.
\end{itemize}


\section{Research Interests}
First-principle computation and simulation;
Computational materials science;
Phonon and phonon interactions; 
Materials thermodynamics and thermal transport;
Condensed matter physics.


\section{Publications}

\begin{enumerate}
\item
\href{https://doi.org/10.1103/PhysRevB.100.014303}
{\underline{Fu, L.}, Kornbluth, M., Cheng, Z., \& Marianetti, C. A. (2019). 
Group theoretical approach to computing phonons and their interactions. 
\textit{Physical Review B}, 100(1), 014303.}
%
\item
\href{https://doi.org/10.1038/s42005-020-00483-2}{
Bryan, M. S., \underline{Fu, L.}, et al. (2020). 
Nonlinear propagating modes beyond the phonons in fluorite-structured crystals.
\textit{Communications Physics}, 3(1), 1-7.}
%
\item
\href{https://doi.org/10.1080/09500839.2020.1833375}{
Ding, X., Yao, T., \underline{Fu, L.}, et al. (2020).
Magnetic, transport and thermal properties of $\delta$-phase UZr$_{2}$.
\textit{Philosophical Magazine Letters}, 1-11.
}
%
\item
\href{https://doi.org/10.1016/j.actamat.2021.116934}{
C.A. Dennett, \dots, \underline{L. Fu}, et al. (2021).
An Integrated Experimental and Computational Investigation of Defect and Microstructural Effects on Thermal Transport in Thorium Dioxide, %
\textit{Acta Mater.}, 213, 116934.
}

\item
\href{https://doi.org/10.1103/PhysRevB.106.014314}{
M. A. Mathis, A. Khanolkar, \underline{L. Fu}, et al. (2022).
Generalized quasiharmonic approximation via space group irreducible derivatives,
\textit{Physical Review B}, 106, 014314.
}
%
% \item
% \href{https://arxiv.org/abs/1903.05253}{
% Wieteska, A., Foutty, B., Guguchia, Z., Flicker, F., Mazel, B., \underline{Fu, L.}, ... \& Pasupathy, A. (2019).
% Uniaxial Strain Tuning of Superconductivity in 2$H$-NbSe$_{2}$.
% \textit{arXiv preprint} arXiv:1903.05253.}
\end{enumerate}


\section{Submitted for Publications}

\begin{enumerate}
\item
\href{https://arxiv.org/abs/2202.11041}{
E. Xiao, H. Ma, M. S. Bryan, \underline{L. Fu}, et al. (2022).
Validating First-Principles Phonon Lifetimes via Inelastic Neutron Scattering,
arXiv:2202.11041.
}
\end{enumerate}


\section{Conferences}

\begin{enumerate}
\item
\href{https://meetings.aps.org/Meeting/MAR18/Event/322388}{
Fu, L., Kornbluth, M., \& Marianetti, C. A. (2018).
An optimal approach to computing phonons and their interactions via finite difference.
APS March Meeting 2018, X29.00006.
}
%
\item
\href{https://meetings.aps.org/Meeting/MAR19/Session/H22.3}{
Fu, L., Kornbluth, M., Cheng, Z., \& Marianetti, C. A. (2019).
An optimal approach to computing phonons and their interactions via finite displacements.
APS March Meeting 2019, H22.00003.
}
%
\item
\href{https://meetings.aps.org/Meeting/MAR20/Session/P44.9}{
Fu, L., Mathis, M., Xiao, E., \& Marianetti, C. A. (2020).
Phonon interactions in rock salt and fluorite structures.
APS March Meeting 2020, P44.00009.
} (\emph{Meeting canceled due to COVID-19 pandemic})
\end{enumerate}

\end{document}
